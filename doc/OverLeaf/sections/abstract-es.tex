\documentclass[../main.tex]{subfiles}

\begin{document}
%% begin abstract format
\makeatletter
\renewenvironment{abstract}{%
    \if@twocolumn
      \section*{Resumen \\}%
    \else %% <- here I've removed \small
    \begin{flushright}
        {\filleft\Huge\bfseries\fontsize{48pt}{12}\selectfont Resumen\vspace{\z@}}%  %% <- here I've added the format
        \end{flushright}
      \quotation
    \fi}
    {\if@twocolumn\else\endquotation\fi}
\makeatother
%% end abstract format
%% begin abstract format
\makeatletter
\renewenvironment{abstract}{%
    \if@twocolumn
      \section*{Resumen \\}%
    \else %% <- here I've removed \small
    \begin{flushright}
        {\filleft\Huge\bfseries\fontsize{48pt}{12}\selectfont Resumen\vspace{\z@}}%  %% <- here I've added the format
        \end{flushright}
      \quotation
    \fi}
    {\if@twocolumn\else\endquotation\fi}
\makeatother
%% end abstract format
\begin{abstract}

Los metadatos aportan información estructurada sobre el contenido de un documento. Esta información no siempre sigue la misma estructura, se encuentra incompleta o es completamente inexistente. Esto supone un problema cuando se quiere clasificar, identificar o ubicar un determinado documento. Con técnicas como la minería de textos (Text Mining) se puede automatizar la tarea de extracción de estos metadatos.

Este proyecto aporta las herramientas necesarias para extraer y generar metadatos de documentos no estructurados. Se emplean técnicas de minería de textos para conseguir dicho propósito. El tipo de documento serán artículos de investigación en inglés y formato PDF. Entre las funcionalidades están: lectura de documentos en formato PDF; análisis y extracción de metadatos; exportar metadatos.

\bfseries{\large{Palabras clave:}} artículo, pdf, metadatos, text mining

\end{abstract}
\end{document}


